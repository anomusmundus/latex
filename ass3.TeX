\documentclass{article}
\usepackage{adjustbox}
\usepackage{graphicx}

\begin{document}

\title{\textbf{Different Forms of Tables}\\\textbf{ (Part-2)}}
\author{Bishal Karmakar}
\date{\today}
\maketitle

\section{Introduction}
This article explains how to use LaTeX to create and customize tables: changing size/spacing, showing caption of tables (below or above), use of positioning parameters, highlighting labels for tables, and changing the direction of the table.

\subsection{Sample 5}
Captions, labels: \\
There are two important commands used as follows:
\begin{enumerate}
  \item \texttt{\textbackslash caption\{\}}: This command is used to make a caption for the table which is placed either above or below the table.
  \item \texttt{\textbackslash label\{\}}: This command is used to refer to a table within a document.
\end{enumerate}

\subsubsection{Positioning parameters and use of \texttt{\textbackslash ref\{\}} command}
The parameter \texttt{h!} passed to the table environment declaration establishes that this table must be placed here, and override LATEX defaults as shown in Table~\ref{tab:sample5}. The positioning parameters that can be passed in include:
\begin{itemize}
  \item \texttt{h}: Will place the table here approximately.
  \item \texttt{t}: Position the table at the top of the page.
  \item \texttt{b}: Position the table at the bottom of the page.
  \item \texttt{p}: Position the table at the top of the next page.
\end{itemize}

\begin{table}[h!]
\centering
\begin{tabular}{||c c c c||}
\hline

Col1 & Col2 & Col3 & Col4 \\
\hline
\hline
1 & 2 & 3 & 4 \\
5 & 6 & 7 & 8 \\
9 & 10 & 11 & 12 \\
13 & 14 & 15 & 16 \\
17 & 18 & 19 & 20 \\
\hline
\end{tabular}
\caption{Table to test 1st positioning parameter (h!)}
\label{tab:sample5}
\end{table}

\subsubsection{To set up the width of table columns}
Here we will see how we can set up the width of columns by specifying a particular width as shown in Table~\ref{tab:sample6}.

\begin{table}[h!]
\centering
\begin{tabular}{|l|l|l|l|}
\hline
Country List & Country Name or Area Name & ISO ALPHA 2 Code & ISO ALPHA 3 Code \\
\hline
Afghanistan & AF & AFG & 004 \\
Aland Islands & AX & ALA & 248 \\
Albania & AL & ALB & 008 \\
\hline
\end{tabular}
\caption{Table to set up the width of table columns}
\label{tab:sample6}
\end{table}

\subsubsection{How to put a caption on top of the Latex table and use of \texttt{\textbackslash vspace\{\}} command}
Here we will show how to put a caption on top of Latex table and adjust the height of the space between the caption and tabular by using the \texttt{\textbackslash vspace\{\}} command. Here the table is placed at the bottom of the page as shown in Table~\ref{tab:sample7}.

\begin{table}[b]
\centering
\vspace{1cm}
\begin{tabular}{|c|c|c|}
\hline
Col1 & Col2 & Col3 \\
\hline
1 & 6 & 87837 \\
2 & 7 & 78 \\
3 & 545 & 778 \\
4 & 545 & 18744 \\
5 & 88 & 788 \\
\hline
\end{tabular}
\caption{Table to test 2nd positioning parameter (b) with the alignment of table columns and caption on top of table}
\label{tab:sample7}
\end{table}

\section{Sample 6}
Reducing the size of an entire Latex table: \\
Here we will see how we are reducing the size of an entire Latex table by using the \texttt{\textbackslash scalebox\{\}} command as shown in Table~\ref{tab:sample8}. We also need to use the \texttt{\textbackslash usepackage\{tcolorbox\}} for implementing the below table.

\begin{table}[h!]
\centering
\scalebox{0.8}{
\begin{tabular}{|c|c|c|c|}
\hline
Col1 & Col2 & Col3 & Col4 \\
\hline
1 & 2 & 3 & 4 \\
5 & 6 & 7 & 8 \\
9 & 10 & 11 & 12 \\
13 & 14 & 15 & 16 \\
17 & 18 & 19 & 20 \\
\hline
\end{tabular}
}
\caption{Table to show a reduction of the size of an entire Latex table}
\label{tab:sample8}
\end{table}

\section{Sample 7}
You can also change the mode to the landscape mode by using the \texttt{\textbackslash usepackage\{adjustbox\}} command. The table here begins with the \texttt{\textbackslash begin\{adjustbox\}} and ends with the \texttt{\textbackslash end\{adjustbox\}}.

The following Table~\ref{tab:sample9} and Table~\ref{tab:sample10} help us to understand this.
\begin{table}[h!]
\centering
\begin{adjustbox}{angle=90}
\begin{tabular}{|c|c|c|}
\hline
heading 1 & heading 2 & heading 3 \\
\hline
$\alpha$ & $\beta$ & $\gamma$ \\
1 & 11.34 & a \\
2 & 10.5 & b \\
3 & 765.5231 & c \\
\hline
\end{tabular}
\end{adjustbox}
\caption{Highlighting rotating of table, angle = 90 degree}
\label{tab:sample9}
\end{table}



For example, if we set the angle to 270 degrees in the same code as used for
the implementation of the previous table (Table 5), the output will look like the
image given below (Table 6):
\begin{table}[h!]
\centering
\begin{adjustbox}{angle=270}
\begin{tabular}{|c|c|c|}
\hline
heading 1 & heading 2 & heading 3 \\
\hline
$\alpha$ & $\beta$ & $\gamma$ \\
1 & 11.34 & a \\
2 & 10.5 & b \\
3 & 765.5231 & c \\
\hline
\end{tabular}
\end{adjustbox}
\caption{Highlighting rotating of table, angle = 270 degree}
\label{tab:sample9}
\end{table}
\end{document}

