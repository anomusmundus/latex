\documentclass{article}
\usepackage{graphicx} % Required for inserting images
\usepackage{multirow}
\usepackage{geometry}
\geometry{a4paper, total = {170mm, 257mm}, left = 20mm, top = 20mm}
\title{Different Forms of Table} \author{Bishal Karmakar} \date{March 12, 2024}
\begin{document}
\maketitle
\section{Introduction}
This article explains how to use LATEX to create and customize tables: changing size\textbackslash 
spacing, combining rows and columns of cells.
\subsection{Sample 1}
Following is the code and result of a simple table created. The tabular environment is the default 
LATEX method to create tables. You must specify a parameter to this environment; here we use \{c 
c c\} which tells LATEX there are three columns and the text inside each one of them must be 
centered.
\begin{center}
\begin{tabular}{c c c}
cell1 & cell2 & cell3\\
cell4 & cell5 & cell6\\
cell7 & cell8 & cell9\\
\end{tabular}
\end{center}
\subsection{Sample 2}
The tabular environment provides additional flexibility for example you can put separator lines in 
between each column. the \textbackslash hline command is used to put a horizontal line on the top 
and bottom of the table. Creating a table with boundaries is demonstrated below:
\begin{center}
\begin{tabular}{ |c|c|c| } 
\hline
cell1 & cell2 & cell3 \\
cell4 & cell5 & cell6 \\
cell7 & cell8 & cell9 \\
\hline
\end{tabular}
\end{center}
\subsection{Sample 3}
We have created a slightly more complex form of table with more rows and columns with a 
applicaion of \textbackslash hline command as per requirement.
\begin{center}
\begin{tabular}{|c c c c|} 
\hline
Col1 & Col2 & Col2 & Col3 \\ [0.5ex] 
\hline\hline
1 & 6 & 87837 & 787 \\
\hline
2 & 7 & 78 & 5415 \\
\hline
3 & 545 & 778 & 7507 \\
\hline
4 & 545 & 18744 & 7560 \\
\hline
5 & 88 & 788 & 6344 \\ [1ex] 
\hline
\end{tabular}
\end{center}
\subsection{Sample 4}
\textbf{Combining rows and columns:}
The command \textbackslash multicolumn and \textbackslash multirow are used to combine rows 
and columns in a table in LATEX.
\subsubsection{Example of multi-column is demonstrated below:}
The number of columns to be combined is displayed in \textbackslash multicolumn \{\}.
\begin{center} \begin{tabular}{ |p{2cm}|p{2cm}|p{2cm}| }
\hline
\multicolumn{3}{|c|}{Books} \\
\hline
Food name& Author &Publication\\
\hline
Book1 & Author1 &P1\\
Book2 & Author2 & P2\\
Book3 & Author3 & P3\\
\hline
\end{tabular} \end{center}
\subsubsection{Example of multi-row is demonstrated below:}
The number of rows to be combined is displayed in \textbackslash multirow \{\}.
\begin{center}
\begin{tabular}{ |c|c|c| } 
\hline
col1 & col2 & col3 \\
\hline
\multirow{3}{4em}{Multiple row} & cell2 & cell3 \\
& cell5 & cell6 \\
& cell8 & cell9 \\
\hline
\end{tabular}
\end{center}
\section{Prepare a record in the form of a table showing the results of a student in 10th standard, 
12th standard, and graduation result serially(SL. no as one column) highlighting the 
boards/University and year of passing in each case in separates columns.}
\begin{center}
 \begin{tabular}{ |p{2cm}|p{2cm}|p{2cm}|p{2cm}| }
 \hline
 \multicolumn{4}{|c|}{Name of the Student} \\
 \hline
 Sl No.& Standard & Board & Percentage\\
 \hline
 & & &\\
 & & &\\
 \hline
\end{tabular}
\end{center}
\end{document}