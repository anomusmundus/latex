\documentclass{article}
\usepackage{graphicx} % Required for inserting images

\title{\huge\textbf{Different Forms Of Tables \\ (Part-1)}}
\author{\large{Bishal Karmakar}}
\date{13 April 2024}




\begin{document}

\maketitle

\section{Introduction}
Images are essential elements in most scientific documents. LATEX provides
several options to handle images and make them look exactly what you need
using the figure environment and the graphicx package. In this article, we
explain how to include images in the most common formats, how to shrink,
enlarge and rotate them, and how to reference them within your document.
\par The figure environment takes care of the numbering and positioning of the
image within the document. In order to include a figure, you must use the
$\backslash$includegraphics command. It takes the image width as an option in brackets
and the path to your image file.

\begin{figure}[h]
   
    \includegraphics[width=5cm,height=4cm]
    {overleaf}
    \centering
    \caption{ Sample 1}
    \label{model1}
\end{figure}
\subsection{Various Formats for Changing the Image Size}
The below figures show how to import a picture. If we want to further specifyhow LATEX should include our image in the document (length, height, etc.),we can pass those settings. The parameters inside the brackets [width=5cm,
\begin{figure}[t]
    \includegraphics[width=\linewidth]
    {overleaf}
    \centering
    \caption{ Sample 2}
    \label{model2}
\end{figure}
height=4cm] define the width and height of the picture, as shown in Figure \ref{model1}.
You can use different units for these parameters.
We can see that the $\backslash$linewidth command is used in the brackets, which means the picture will be scaled to fit the width of the document, as shown in Figure \ref{model2}.
\newpage
The command $\backslash$includegraphics[scale=1.5]\{overleaf\} will include the image overleaf-logo in the document, the extra parameter scale=1.5 will do exactly that, scale the image 1.5 of its real size as shown in Figure \ref{model3}.
\begin{figure}[h]
    \includegraphics[scale=1.5]
    {overleaf}
    \centering
    \caption{ Sample 3}
    \label{model3}
\end{figure}
If only the width parameter is passed, the height will be scaled to keep the aspect ratio, as shown in Figure \ref{model4}. The length of units can also be relative to some elements in the document. If you want, for instance, make a picture the same width as the text, as shown in Figure \ref{model5}.
\begin{figure}[h]
    \centering
    \includegraphics[width=7cm, height=5cm]{overleaf}
    \caption{Sample 4}
    \label{model4}
\end{figure}
\newpage
\begin{figure}[h]
    \centering
    \includegraphics[width=15cm, height=9cm]{overleaf}
    \caption{Sample 5}
    \label{model5}
\end{figure}
There is another common option when including a picture within your document to rotate it. It can easily be accomplished in LATEX. The parameter
angle=45 rotates the picture 45 degrees counter-clockwise, as shown in Figure 6. To rotate the picture clockwise, use a negative number, as shown in Figure7.
\newpage
\begin{figure}[h]
    \centering
    \includegraphics[width=3cm, height=2cm,angle=45]{overleaf}
    \caption{Sample 6}
    \label{model6}
\end{figure}

\begin{figure}[h]
    \centering
    \includegraphics[width=3cm, height=2cm,angle=-45]{overleaf}
    \caption{Sample 7}
    \label{model7}
\end{figure}
If we want to reduce the size of the picture, as shown in Figure \ref{model8}, we can also implement it with the help of changing the width of textwidth.

\begin{figure}[h]
    \centering
    \includegraphics[width=2cm, height=3cm]{overleaf}
    \caption{Sample 8}
    \label{model8}
\end{figure}
\end{document}