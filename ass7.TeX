\documentclass{article}
\usepackage{graphicx} % Required for inserting images
\usepackage{amsmath}

\title{Mathematical Equations in Latex \\ (Part-1)}
\author{Bishal Karmakar}
\date{23 April 2024}

\begin{document}

\maketitle

There are two major modes of typesetting math in LaTeX: one is embedding the math directly into your text by encapsulating your formula in dollar signs, and the other is using a predefined math environment.

\section{Mathematical modes}
LaTeX allows two writing modes for mathematical expressions: the inline math mode and the display math mode.
\begin{itemize}
    \item \textbf{Inline math mode} is used to write formulas that are part of a paragraph.
    \item \textbf{Display math mode} is used to write expressions that are not part of a paragraph and are therefore put on separate lines.
\end{itemize}

\subsection{Inline math mode}
You can use any of these “delimiters” to typeset your math in inline mode. They all work, and the choice is a matter of taste.
\begin{enumerate}
    \item Standard LaTeX practice is to write inline math by enclosing it between \verb|\(...\)|:
    \begin{verbatim}
    In physics, the mass-energy equivalence is stated by the equation
    \(E = mc^2\), discovered in 1905 by Albert Einstein.
    \end{verbatim}
    
    \item Instead of writing (enclosing) inline math between \verb|\(...\)|, you can use \verb|$...$| to achieve the same result:
    \begin{verbatim}
    In physics, the mass-energy equivalence is stated by the equation
    $E = mc^2$, discovered in 1905 by Albert Einstein.
    \end{verbatim}
    
    \item Or, you can use \verb|\begin{math}...\end{math}|:
    \begin{verbatim}
    In physics, the mass-energy equivalence is stated by the equation
    \begin{math}E = mc^2\end{math}, discovered in 1905 by Albert Einstein.
    \end{verbatim}
\end{enumerate}




\subsection{Display math mode}
Display math mode has two versions which produce numbered or unnumbered equations. Let’s look at a basic example:
The mass-energy equivalence is described by the famous equation:
\[E = mc^2\]
discovered in 1905 by Albert Einstein. In natural units (\(c = 1\)), the formula expresses the identity
\[E = m\]

\section{The equation and align environment}
The most useful math environments are the \texttt{equation} environment for typesetting single equations and the \texttt{align} environment for multiple equations and automatic alignment:
\begin{align}
1 + 2 &= 3 \\
1 &= 3 - 2 \\
1 + 2 &= 3 \\
1 &= 3 - 2
\end{align}
The \texttt{align} environment will align the equations at the ampersand (\&). Single equations have to be separated by a linebreak. There is no alignment when using the simple \texttt{equation} environment. Furthermore, it is not even possible to enter two equations in that environment; it will result in a compilation error. The asterisk (e.g., \texttt{equation*}) only indicates that we don’t want the equations to be numbered.

\section{Fractions and more}
LaTeX is capable of displaying any mathematical notation. It’s possible to typeset integrals, fractions, and more. Every command has a specific syntax to use. We will demonstrate some of the most common LaTeX math features. We have also highlighted the possibility of combining various commands to create more sophisticated expressions.




\[
f(x) = x^2 \\
g(x) = \frac{1}{x} \\
F(x) = \int_{a}^{b} \frac{1}{3} x^3 \, dx \\
gm(x) = \frac{1}{\sqrt{x}}
\]

The more complex the expression, the more error-prone this is; it’s important to take care of opening and closing the braces {}. It can take a long time to debug such errors.

More examples:
\begin{enumerate}
    \item 
    \[
    x^3 + y^3 = 9 \quad (2)
    \]
    \[
    x^2 + 2x + 4 = 0 \quad (3)
    \]
    \[
    y^2 + 4y = 5 \quad (4)
    \]
    \[
    v = u + at \quad (5)
    \]
    \[
    v^2 = u^2 + 2as \quad (6)
    \]
    \[
    s = ut + \frac{1}{2} at^2 \quad (7)
    \]
    \[
    \cos^2 \theta + \sin^2 \theta = 1 \quad (8)
    \]
    \[
    \cos^2 \theta = 1 - 2\sin^2 \theta \quad (9)
    \]
    \[
    \log a = \log b \quad (10)
    \]
    \[
    \log a + \log b = \log c + \log d \quad (11)
    \]
    which can also be written as
    \[
    \log ab = \log cd \quad (12)
    \]
    
    \item The BSRCS between two nodes \(i\) and \(j\) is expressed as in Equation (13):
    \[
    BSRCS_{ij} = \frac{SR_i}{(\sum_{k} SR_k)} - SR_i \times \frac{SR_j}{(\sum_{k} SR_k)} - SR_j \quad (13)
    \]
\end{enumerate}

\end{document}
