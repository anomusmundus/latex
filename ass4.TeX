\documentclass{article}
\usepackage{adjustbox}
\usepackage{graphicx}
% \usepackage{geometry}
\usepackage{multirow}
\usepackage{tabularx}
\usepackage{float}
% \geometry{a4paper, total = {170mm, 257mm}, left = 16mm, top = 0mm}
\title{Different Forms of Table} \author{Bishal Karmakar} 
\begin{document}

\title{\textbf{Different Forms of Tables}\\\textbf{ (Part-3)}}
\author{Bishal Karmakar}
\date{2 April 2024}
\maketitle
\listoftables{}

\section{List of Tables}
To create a list of tables use the \textbackslash listoftables{} command. The caption of each table will be used to generate this list.

\begin{table}[h!]
\centering
\begin{tabular}{||c c c c||}
\hline

Col1 & Col2 & Col3 & Col4 \\
\hline
\hline
1 & 6 & 87837 & 787 \\
2 & 7 & 78 & 5415 \\
3 & 545 & 778 & 7507 \\
4 & 545 & 18744 & 7560 \\
5 & 88 & 788 & 6344 \\
\hline
\end{tabular}
\caption{This is the caption for the first table.}
\label{tab:sample5}
\end{table}

\begin{table}[h!]
\centering
\begin{tabular}{||c c c c||}
\hline

Col1 & Col2 & Col3 & Col4 \\
\hline
\hline
4 & 545 & 18744 & 7560 \\
5 & 88 & 788 & 6344 \\
\hline
\end{tabular}
\caption{This is the caption for the second table.}
\label{tab:sample5}
\end{table}


\begin{table}
\centering
\begin{tabularx}{\textwidth}{|X|X|X|X|X|X|X|X|X|X|}
\hline
\multicolumn{10}{|c|}{Demo of a Complex Form of Table} \\
\hline
Weights & $\tau$ & $E^{(C)}$ & $T^{(D)}$ & $\beta^{(Avg)}$ & $F^{(50+100)}$ & Ct & $W^{(C)}$ & $Bo^{(\alpha)}$ & $Bo^{(\gamma)}$\\
$(\alpha)$ & $(w_{2}^{(nl)})$ & $(w_{4}^{(nl)})$ & $(w_{6}^{(nl)})$ & $(w_{8}^{(nl)})$ & $(w_{10}^{(nl)})$ & $(w_{12}^{(nl)})$ & $(w_{14}^{(nl)})$ & $(w_{16}^{(nl)})$ & $(w_{18}^{(nl)})$\\
\hline
+0.01 & 0.081 & 0.131 & 0.013 & 0.132 & 0.150 & 0.122 & -0.074 & 0.014 & 0.002 \\
-0.01 & 0.082 & 0.138 & 0.007 & 0.139 & 0.159 & 0.128 & -0.091 & 0.007 & -0.005 \\
+0.03 & 0.080 & 0.126 & 0.019 & 0.126 & 0.142 & 0.117 & -0.060 & 0.019 & 0.009 \\
\hline
\end{tabularx}
\caption{Creating Complex Tables-1.}
\label{tab:sample5}
\end{table}



\section{Creating Complex Tables}
Here we will see how to create complex forms of tables by incorporating various mathematical symbolic representations like $\tau$ , $\beta$, etc. Furthermore, we will see how to use both subscripts and superscripts involving exponents, indexes, and some special operators in the same mathematical expressions, such as $(w_{8}^{(nl)})$, $(w_{16}^{(nl)})$. Table 3 displays all of the types.


\section{Assignment to be done}
The following Table 4 is to be executed as an assignment.

\begin{table}[h!]
\centering
\scalebox{0.8}{

\begin{tabular}{|*{18}{c|}}  % repeats {c|} 18 times
\hline
\multicolumn{9}{|c}{k-means clustering} & \multicolumn{9}{|c|}{Fuzzy c-means clustering} \\ \hline
\multicolumn{3}{|c}{50 clusters} & \multicolumn{3}{|c}{60 clusters} & \multicolumn{3}{|c}{70 clusters} & 
\multicolumn{3}{|c}{50 clusters} & \multicolumn{3}{|c}{60 clusters} & \multicolumn{3}{|c|}{70 clusters} \\ \hline 
CJ & HT & SVD &CJ & HT & SVD &CJ & HT & SVD &CJ & HT & SVD &CJ & HT & SVD &CJ & HT & SVD \\ \hline
 & & & & & & & & & & & & & & & & &  \\ \hline
\end{tabular}
}
\caption{Creating Complex Tables-2.}
\label{tab:sample5}
\end{table}




\end{document}